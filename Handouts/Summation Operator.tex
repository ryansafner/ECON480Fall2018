\documentclass{article}
\usepackage{amssymb, amsmath, amsthm}
\usepackage[margin=1in]{geometry}

\title{Econometrics: The Summation Operator}
\author{Ryan Safner}
\date{Fall 2018}

\begin{document}
	\maketitle
	
\section{Definition}
Many elementary propositions in econometrics (and statistics) involve the use of the sums of numbers. Mathematicians often use the summation operator (the greek letter $\Sigma$--``sigma'') as a shorthand, rather than writing everything out the long way. It will be worth your time to understand the summation operator, and some of its properties, and how these can provide shortcuts to proving more advanced theorems in econometrics. 

Let $X$ be a random variable from which a sample of $n$ observations is observed, so we have a sequence $\{x_1, x_2,...,x_n\}$ i.e. $x_i, $ for $i=1,2,...,n$. Then the total sum of the observations $(x_1+x_2+...+x_n)$ can be represented as:
\begin{equation*}
\sum_{i=1}^n x_i = x_1+x_2+...+x_n
\end{equation*}
\begin{itemize}
	\item The term beneath $\Sigma$ is known as the ``index,'' which tells us where to begin our adding (at the 1\textsuperscript{st} individual $x$ term, $x_1$). 
	\begin{itemize}
		\item Note other letters, such as $j$, or $k$ may be used (especially if $i$ is defined elsewhere)
	\end{itemize} 
	\item The term above $\Sigma$ is the total number of $x$ terms we should add ($n$)
	\item Essentially, read $\displaystyle \sum_{i=1}^{n} x_i$ as ``add up all the individual $x$ observations from the 1\textsuperscript{st} $(x_1)$ to the final $n$\textsuperscript{th} ($x_n$).''  	
\end{itemize}

\section{Useful Properties}

\textbf{Rule 1}: The summation of a constant $k$ times a random variable $X_i$ is equal to the constant times the summation of that random variable: 
\begin{equation*}
\sum_{i=1}^n kX_i = k \sum^n_{i=1} X_i
\end{equation*}

\begin{proof}
\begin{align*}
\sum_{i=1}^n kX_i &= k x_1 + kx_2 +...+ kx_n\\
&= k(x_1+x_2+...x_n)\\
&= k\sum_{i=1}^n X_i. \\
\end{align*}
\end{proof}

\clearpage 

\textbf{Rule 2}: The summation of a sum of two random variables is equal to the sum of their summations: 

\begin{equation*}
\sum_{i=1}^n (X_i+Y_i) = \sum_{i=1}^n X_i + \sum_{i=1}^n Y_i
\end{equation*}

\begin{proof}
\begin{align*}
\sum_{i=1}^n (X_i+Y_i) &=(X_1+Y_1) + (X_2+Y_2) + ... (X_n+Y_n)\\
&=(X_1+X_2+...+X_n) + (Y_1+Y_2+...+Y_n)\\
&=\sum_{i=1}^n (X_i+Y_i)\\
\end{align*}
\end{proof}

\textbf{Rule 3}: The summation of constant over $n$ observations is the product of the constant and $n$: 

\begin{equation*}
\sum_{i=1}^n k = nk
\end{equation*}

\begin{proof}

\begin{equation*}
\sum_{i=1}^n k = \underbrace{k + k + ... + k}_{n \text{ times}} = nk
\end{equation*}

\end{proof}

\textbf{Combining these 3 rules:} for the sum of a linear combination of a random variable ($a+bX$)

\begin{equation*}
\sum_{i=1}^n (a+bX_i) = na+b\sum_{i=1}^n X_i
\end{equation*}

\begin{proof}

Left to you as an exercise! 
\end{proof}


\clearpage 

\section{Advanced: Useful Properties for Regression}

There are some additional properties of summations that may not be immediately obvious, but will be quite essential in proving properties of linear regressions.

Using the properties above, we can describe the \textbf{mean}, \textbf{variance}, and \textbf{covariance} of random variables.\footnote{For more beyond the mere definition, see the handout on \textbf{Covariance and Correlation}}

First, define the mean of  a sequence $\{X_i: i=1,...,n\}$ and $\{Y_i: i=1,...,n\}$ as: 

\begin{equation*}
\bar{X}=\frac{1}{n}\sum^n_{i=1}X_i
\end{equation*}

Second, the variance of  $X$ is:

\begin{equation*}
var(X)=\frac{1}{n}\sum^n_{i=1}(X_i-\bar{X})^2
\end{equation*}

Third, the covariance of $X$ and $Y$ is:

\begin{equation*}
cov(X,Y)=\frac{1}{n}\sum^n_{i=1}(X_i-\bar{X})(Y_i-\bar{Y})
\end{equation*}


\textbf{Rule 4}: The sum of the deviations of observations of $X_i$ from its mean ($\bar{X}$) is 0:

\begin{equation}
	\sum^n_{i=1} (X_i-\bar{X})=0
\end{equation}

\begin{proof}
\begin{align*}
	\sum^n_{i=1} (X_i-\bar{x}) &=\sum^n_{i=1}X_i-\sum^n_{i=1}\bar{X} && \\
	&=\sum^n_{i=1}X_i-n\bar{X} && \text{Since $\bar{x}$ is a constant}\\
	&=n\underbrace{\frac{\displaystyle\sum^n_{i=1}X_i}{n}}_{\bar{X}}-n\bar{X} && \text{Multiply the first term by }\frac{n}{n}=1\\ 
	&=n\bar{X}-n\bar{X}&& \text{By the definition of the mean }\bar{X}\\
	&=0 &&  \\
\end{align*}
\end{proof}

\clearpage 

\textbf{Rule 5}: The squared deviations of $X$ are equal to the product of $X$ times its deviations: 

\begin{equation*}
\sum^n_{i=1} (X_i-\bar{X})^2=\sum^n_{i=1} X_i(X_i-\bar{X})	
\end{equation*}

\begin{proof}
	
\begin{align*}
	\sum^n_{i=1} (X_i-\bar{X})^2&=\sum^n_{i=1} (X_i-\bar{X})(X_i-\bar{X})	&& \text{Expanding the square}\\
	&=\sum^n_{i=1} X_i(X_i-\bar{X})-\sum^n_{i=1}\bar{X}(X_i-\bar{X}) && \text{Breaking apart the first term} \\
	&=\sum^n_{i=1} X_i(X_i-\bar{X})-\bar{X}\sum^n_{i=1}(X_i-\bar{X}) && \text{Since }\bar{X} \text{ is constant, not depending on } i's\\
	&=\sum^n_{i=1} X_i(X_i-\bar{X})-\bar{X}(0) && \text{From rule 4} \\
	&=\sum^n_{i=1} X_i(X_i-\bar{X}) && \text{Remainder after multiplying by 0}\\
\end{align*}

\end{proof}

\textbf{Rule 6}: The following summations involving $X$ and $Y$ are equivalent: 

\begin{equation*}
\sum^n_{i=1} Y_i(X_i-\bar{X})=\sum^n_{i=1}X_i(Y_i-\bar{Y})=\sum^n_{i=1} (X_i-\bar{X})(Y_i-\bar{Y})	
\end{equation*}

\begin{proof}

\begin{align*}
	\sum^n_{i=1} (X_i-\bar{X})(Y_i-\bar{Y}) &=\sum^n_{i=1} Y_i(X_i-\bar{X})-\sum^n_{i=1}\bar{Y}(X_i-\bar{X}) && \text{Breaking apart the second term} \\
	&=\sum^n_{i=1} Y_i(X_i-\bar{X})-\bar{Y}(0) && \text{From rule 4}\\
	&=\sum^n_{i=1} Y_i(X_i-\bar{X}) && \text{Remainder after multiplying by 0}\\
\end{align*}

equivalently:

\begin{align*}
	\sum^n_{i=1} (X_i-\bar{X})(Y_i-\bar{Y}) &=\sum^n_{i=1} X_i(Y_i-\bar{Y})-\sum^n_{i=1}\bar{X}(Y_i-\bar{Y}) && \text{Breaking apart the first term} \\
	&=\sum^n_{i=1} X_i(Y_i-\bar{Y})-\bar{X}(0) && \text{From rule 4}\\
	&=\sum^n_{i=1} X_i(Y_i-\bar{Y}) && \text{Remainder after multiplying by 0}\\
\end{align*}

\end{proof}

\end{document}